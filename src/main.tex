\documentclass{article}

% Packages
\usepackage{amsmath} % For mathematical symbols and equations
\usepackage{graphicx} % For including images
\usepackage{cite} % For citations
\usepackage{url} % For URLs
\usepackage{hyperref} % For clickable links
\usepackage{epigraph} % For quotes

\setlength\epigraphwidth{.6\textwidth}
\renewcommand{\epigraphflush}{flushright}

% Document information
\title{Turing Completeness of Neural Networks}
\author{D'Ambrosi Denis \\ \small \texttt{dambrosi.denis@spes.uniud.it} \\ \small \texttt{147681}}
\date{\today}

\begin{document}

\maketitle

\begin{abstract}
    Neural networks are a powerful machine learning tool that can be
    used for a wide range of tasks, from image classification to natural
    language processing. At their core, neural networks are composed
    of interconnected nodes that process and transmit information. But
    are they Turing complete? In this paper, we will explore the Turing
    completeness of neural networks and the implications of this
    property for their use in computation.
\end{abstract}

\section{Introduction}

\epigraph{\textit{A neural network is the second best way to solve any problem. The best way is to actually understand the problem.}}{\textit{Unknown}}

% Bibliography
\bibliographystyle{IEEEtran} % Choose a bibliography style
\bibliography{references} % Specify the bibliography file

\end{document}